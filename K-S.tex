\documentclass[a4paper,10pt,english]{article}% norsk istedefor english
%\usepackage{ucs}
%\usepackage[utf8x]{inputenc}%utf8x/ UiO utf-8
\usepackage[latin1]{inputenc}%utf8x/ UiO utf-8
\usepackage{babel}
\usepackage{fontenc}%[T1]
\usepackage{graphicx,subfigure,epsfig}%,uioforside},babel hvis norsk
\usepackage{amssymb,psfrag,amsmath,wick,axodraw}
%\usepackage[novbox]{pdfsync}

\setlength{\parindent}{0in} %Ingen indent

\include{mineshortcuts}

\begin{document}
\section{Introduction}
Modern density functional theory developed from the Thomas-Fermi approach, where
it was suggested that the energy can be written exclusively in terms of the 
electronic density. It was not until Hohenberg and Kohn in 1964 proved a theorem
for this relation between the energy and the electronic density that DFT became
a widely used tool in many-body computations.

\section{The Hohenberg-Kohn Theorem}
The theorem states that the external potential is univocally determined by the 
electron density, besides an additive constant.

The proof is in two parts, first we may prove that the opposite holds, ie. the
external potential is not univocally determined by the density. This means that
there should exist two different external potential $v_1$ and $v_2,$ where
$v_1\neq v_2 +cr,$ but there ground state density $\rho$ is the same. Let us
then denote the Hamiltonians for the two different potentials $H_1=T+U^{ext}_1+V_{ee}$ and $H_2=T+U^{ext}_2+V_{ee}.$ Then we know that the ground state
energy for $H_1$ is given by 
\begin{equation*}
        \begin{split}
                &E_0=\braopket{\Phi^1_0}{H_1}{\Phi^1_{0}}<\braopket{\Phi^2_0}{H_1}{\Phi^2_0}=\\
                &\braopket{\Phi^2}{H_2}{\Phi_2}+\braopket{\Phi_2}{H_1-H_2}{\Phi_2}\\
                &=E_0'+\int\rho(r)(v_1-v_2)d^3r,
        \end{split}
\end{equation*}
 and the ground state energy for $H_2$ is given by
\begin{equation*}
        \begin{split}
                &E_0'=\braopket{\Phi^2}{H_2}{\Phi_2}<\braopket{\Phi^1_0}{H_1}{\Phi^1_{0}}+\braopket{\Phi_1}{H_2-H_1}{\Phi_1}\\
                &=E_0+\int\rho(r)(v_2-v_1)d^3r
        \end{split}
        \label{sc1eq:1}
\end{equation*}
If we now add the two ground state energies together we see a contradicting 
statement
\begin{equation*}
       E_0+E_0'<E_0+E_0'.
        \label{sqeq:contr}
\end{equation*}
Thus we see that one density cannot determine two different densities, differing
by more than an additive constant.\\

We note that since the density $\rho(r)$ univocally determines den external potential $v$, it also determines the ground state wave function $\Phi_0$.\\

The second theorem of Hohenberg-Kohn states that if $\tilde \rho(r)$ is a non-negative density and normalized to the number of particles $N$. Then the 
variational energy $E_V[\rho]$ defined by
\begin{equation}
  E_V[\rho]=F[\rho]+\int\rho(r)v_{ext}(r)dr
  \label{s1eq:energyv}
\end{equation}
with the functional 
\begin{equation}
  F[\rho]=\braopket{\Phi[\rho]}{T+U_{ee}}{\Phi[\rho]}.
  \label{s1eq:Fu}
\end{equation}
If the wavefunction $\Phi[\rho]$ is the ground state of a potential with $\rho$ as its density, ie. if $E_0=E_V$ then 
\begin{equation*}
  E_0 < E_V[\tilde \rho].
\end{equation*} 
As a proof we see that the density $\rho$ fulfills the condition 
\begin{equation}
  \delta\big\{E_V[\rho]-\mu\big(\int\rho(r)dr-N\big)\big\}=0,
  \label{s1eq:variationl}
\end{equation}
thus 
\begin{equation}
  \begin{split}
	&\braopket{\Phi[\tilde \rho]}{H}{\Phi[\tilde \rho]}=F[\tilde \rho]+\int\tilde \rho(r)v_{ext}(r)dr\\
	&=E_{V}[\tilde \rho]>E_V[\rho]=E_0=\braopket{\Phi[\rho]}{H}{\Phi[\rho]}.
  \end{split}
  \label{s1eq:proof2}
\end{equation}
We see that the knowledge of $F[\rho]$ implies the knowledge of the solution of
the full many-body Schr\"odinger equation and that it is a universal functional
which does not depend on the external potential $v_{ext}$.

\section{Kohn-Sham equations}

Kohn and Sham \cite{Kohn1965}  suggested to use a system of non-interacting electrons, 
for which a Slater determinant can describe exactly, which gives the same
electron density as the interacting system. Thus the kinetic energy term is described by 
\begin{equation}
  T=\sum_i^N \bra{\varphi_i}\frac{\nabla^2}{2}\ket{\varphi_i}.
  \label{s2ksT}
\end{equation}
The kinetic energy operator will of course not give the exact kinetic energy of
the interacting system, which follows from the observation that the wavefunction
of the true problem is not described purely by a Slater determinant.\\

We describe the Hamiltonian of the non-interacting system in the following way
\begin{equation}
  H_{KS}=\sum_{i}^N\big[-\frac{\nabla^2}{2}+v_{KS}\big],
  \label{s2eq:Hks}
\end{equation}
where $N$ is the number of electrons. Because of the absence of 
electron-electron interactions in the Hamiltonian, its eigenstates can be 
expressed in the form of Slater determinants, with density
\begin{equation}
  \rho(r)=\sum_i^{N/2}2|\varphi_i(r)|^2,
  \label{s2eq:KSdens}
\end{equation}
for a closed shell system. The single particle orbitals are the solution of the
Kohn-Sham Hamiltionian
\begin{equation}
  H_{KS}\varphi_i=\frac{\nabla^2}{2}\varphi_i+v_{KS}\varphi_i=\varepsilon_i\varphi_i.
  \label{s2eq:kseigfe}
\end{equation}

The Kohn-Sham funcional defined by
\begin{equation}
  %\begin{split}
	E_{KS}=T_R[\rho]+\frac{1}{2}\int\frac{\rho(r)\rho(r')}{|r-r'|}drdr'+E_{XC}[\rho]+\int\rho(r)v_{ext}(r)dr,
  %\end{split}
  \label{s2eq:KSenergy}
\end{equation}
is used for the determination of the Kohn-Sham Hamiltonian in Eq.\eqref{s2eq:Hks}.
To find the ground state density, the variational principle in Eq. \eqref{s1eq:variationl} is applied to Eq.\eqref{s2eq:KSenergy} which yields
\begin{equation}
  \frac{\delta T_R}{\delta \rho(r)}+v_{ext}(r)+\int\frac{\rho(r')}{|r-r'|}dr'+\frac{\delta E_{XC}[\rho(r)]}{\delta\rho(r)}=\mu.
  \label{s2eq:varappl}
\end{equation}
By comparing this equation with the energy funcitonal for the non-interacting
system, 
\begin{equation}
  E_{NI}[\rho]=T_R[\rho]+\int\rho(r)v_{KS}(r)dr,
  \label{s2eq:noninterfunc}
\end{equation}
we may identify the the Kohn-Sham potential,$v_{KS}$ with
\begin{equation}
  v_{KS}=v_{ext}(r)+\int\frac{\rho(r')}{|r-r'|}dr'+v_{XC}[\rho]
  \label{s2eq:vKS}
\end{equation}
with 
\begin{equation}
  v_{KS}=\frac{\delta E_{XC}[\rho]}{\delta \rho(r)}.
  \label{s2eq:vXC}
\end{equation}
Thus our Schr\"odinger equation Eq.\eqref{s2eq:kseigfe} stated again
\begin{equation*}
  H_{KS}\varphi_i=\frac{\nabla^2}{2}\varphi_i+v_{KS}\varphi_i=\varepsilon_i\varphi_i.
%  \label{s2eq:kseigfe}
\end{equation*}
has to be solved self consistently, since the Hamiltonian depends on the density in Eq. \eqref{s2eq:KSdens}. The total energy of the interacting system is given by
\begin{equation}
  E[\rho]=\sum_i^{N_s}\varepsilon_i-\frac{1}{2}\int\frac{\rho(r)\rho(')}{|r-r'|}drdr'+\int\rho(r)\big(\varepsilon_{XC}[\rho]-v_{XC}[\rho]\big).
  \label{s2eq:totalenergy}
\end{equation}
Here $\varepsilon_{XC}$ is the exchange correlation energy density dealt with in a later section. The exchange correlation energy includes correlation from
the kinetic energy as well as from the electron-electron interaction wihich the
non-interacting system does not explain correctly, because of the fact that
the wavefunction cannot be described accurately by a Slater determinant.



\bibliographystyle{h-physrev3}
\bibliography{DFT.bib}
\end{document}

\documentclass[a4paper,10pt,english]{article}% norsk istedefor english
%\usepackage{ucs}
%\usepackage[utf8x]{inputenc}%utf8x/ UiO utf-8
\usepackage[latin1]{inputenc}%utf8x/ UiO utf-8
\usepackage{babel}
\usepackage{fontenc}%[T1]
\usepackage{graphicx,subfigure,epsfig}%,uioforside},babel hvis norsk
\usepackage{amssymb,psfrag,amsmath,wick,axodraw}
%\usepackage[novbox]{pdfsync}

\setlength{\parindent}{0in} %Ingen indent

\include{mineshortcuts}

\begin{document}
\section{Introduction}
Modern density functional theory developed from the Thomas-Fermi approach, where
it was suggested that the energy can be written exclusively in terms of the 
electronic density. It was not until Hohenberg and Kohn in 1964 proved a theorem
for this relation between the energy and the electronic density that DFT became
a widely used tool in many-body computations.

\section{The Hohenberg-Kohn Theorem}
The theorem states that the external potential is univocally determined by the 
electron density, besides an additive constant.

The proof is in two parts, first we may prove that the opposite holds, ie. the
external potential is not univocally determined by the density. This means that
there should exist two different external potential $v_1$ and $v_2,$ where
$v_1\neq v_2 +cr,$ but there ground state density $\rho$ is the same. Let us
then denote the Hamiltonians for the two different potentials $H_1=T+U^{ext}_1+V_{ee}$ and $H_2=T+U^{ext}_2+V_{ee}.$ Then we know that the ground state
energy for $H_1$ is given by 
\begin{equation*}
        \begin{split}
                &E_0=\braopket{\Phi^1_0}{H_1}{\Phi^1_{0}}<\braopket{\Phi^2_0}{H_1}{\Phi^2_0}=\\
                &\braopket{\Phi^2}{H_2}{\Phi_2}+\braopket{\Phi_2}{H_1-H_2}{\Phi_2}\\
                &=E_0'+\int\rho(r)(v_1-v_2)d^3r,
        \end{split}
\end{equation*}
 and the ground state energy for $H_2$ is given by
\begin{equation*}
        \begin{split}
                &E_0'=\braopket{\Phi^2}{H_2}{\Phi_2}<\braopket{\Phi^1_0}{H_1}{\Phi^1_{0}}+\braopket{\Phi_1}{H_2-H_1}{\Phi_1}\\
                &=E_0+\int\rho(r)(v_2-v_1)d^3r
        \end{split}
        \label{sc1eq:1}
\end{equation*}
IF we now add the two ground state energies together we see a contradicting 
statement
\begin{equation*}
       E_0+E-0'<E_0+E_0'.
        \label{sqeq:contr}
\end{equation*}
Thus we see that one density cannot determine two different densities, differing
by more thant an additive constant.






\end{document}
